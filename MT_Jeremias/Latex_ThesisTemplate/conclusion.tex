%!TEX root = thesis.tex
\chapter{Conclusion}
\label{ch:conclusion}

In this thesis, a simulation methodology to evaluate a dynamic photovoltaic shading system is presented, combining both electricity generation, and the energy demand of the building. It is then coupled with a post processing python script to determine the optimum system configuration for control. The methodology can be applied to evaluate different PV system geometries, building systems, building typologies and climates.

The dynamic PV integrated shading system has clear advantages to a static system as it can adapt itself to the external environmental conditions. This enables it to orientate itself to the most energy efficient position. The optimum orientation however, strongly depends on the general efficiency of the building. Decreasing the efficiency of the heating, cooling or lighting systems will give higher preference for configurations optimised for building thermal management through adaptive shading, than for PV electricity production.

%The use of LED lights, for example, reduces the weighting of the lighting energy demand. This would result in closed configurations optimised for cooling to over-ride the open positions. 

This work ultimately presents a methodology for the planning and optimisation of sophisticated adaptive BIPV systems. Future work will use this methodology to determine the environments and building typologies that could benefit from adaptive BIPV systems. 